
\mode<presentation>
{
  %\usetheme[secheader]{Madrid} % nice (once my coloroverrides are sorted out)
  %\usetheme{AnnArbor}	% nice!	
  %\usetheme{Malmoe}	% nice!	
  \usetheme{Warsaw}	% nice!	

  %\usetheme[secheader]{Boadilla} % ok
  %\usecolortheme{whale}
  %\usecolortheme{orchid}
}

% If you wish to uncover everything in a step-wise fashion, uncomment the following command: 
%\beamerdefaultoverlayspecification{<+->}

% Delete this, if you do not want the table of contents to pop up at
% the beginning of each subsection (or section)
% edd: Does not work in handout mode, and we have too many section/subsections
%\AtBeginSection[]{%
%  \begin{frame}<beamer>%
%    %\tiny
%    \frametitle{Outline}%
%    \tableofcontents[currentsection]%
%  \end{frame}
%}

%\AtBeginSubsection[]{%
%   \begin{frame}
%     \frametitle{Outline}
%     \tableofcontents[currentsection,currentsubsection]
%   \end{frame}
%}

% Delete this, if you do not want the table of contents to pop up at
% the beginning of each subsection (or section)
% edd: Does not work in handout mode, and we have too many section/subsections
\AtBeginSection[]{%
  \begin{frame}<beamer>%
    \frametitle{Outline}%
    %\tableofcontents[currentsection]%
    \tableofcontents[currentsection,hideothersubsections]%
  \end{frame}
}
 
%\AtBeginSection[] % Do nothing for \section* 
%{ \begin{frame}<beamer>
%    \frametitle{Outline}
%    {%\footnotesize 
%      \begin{columns}
%        \begin{column}{0.5\textwidth}
%          \tableofcontents[currentsection,sections={1-2}]
%        \end{column}
%        \begin{column}{0.5\textwidth}
%          \tableofcontents[currentsection,sections={3-4}]
%        \end{column}
%      \end{columns}
%    } 
%  \end{frame} 
%}

\newcommand{\MedSkip}{\medskip \par} % add \pause if desired 
\newcommand{\SmallSkip}{\smallskip} % add \pause if desired 

\usepackage[english]{babel}   	% or whatever
\usepackage[latin1]{inputenc}	% or whatever
\usepackage{times}
\usepackage[T1]{fontenc}	% Or whatever. Note that the encoding and the 
				% font should match. If T1 does not look
                                % nice, try deleting the line with the fontenc.
%\usepackage{highlight}

\usepackage{booktabs}

\usepackage{listings}
\lstset{ %
  language=R,                   % choose the language of the code
  basicstyle=\scriptsize,       % the size of the fonts that are used for the code
  numbers=left,                 % where to put the line-numbers
  numberstyle=\tiny,            % the size of the fonts that are used for the line-numbers
  stepnumber=1,                 % the step between two line-numbers. If it's 1 each line will be numbered
  numbersep=5pt,                % how far the line-numbers are from the code
  backgroundcolor=\color{white},% choose the background color. You must add \usepackage{color}
  showspaces=false,             % show spaces adding particular underscores
  showstringspaces=false,       % underline spaces within strings
  showtabs=false,               % show tabs within strings adding particular underscores
  frame=single,                 % adds a frame around the code
  tabsize=2,                    % sets default tabsize to 2 spaces
  captionpos=b,                 % sets the caption-position to bottom
  breaklines=true,              % sets automatic line breaking
  breakatwhitespace=false,      % sets if automatic breaks should only happen at whitespace
  escapeinside={\%*}{*)}        % if you want to add a comment within your code
}

\hypersetup{                  		% beamer colors taken from elsewhere
  hyperindex,%				% works with the beetle colour scheme
  colorlinks,%
  linktocpage,%
  plainpages=true,%
  linkcolor=myOrange,%
  citecolor=myDarkGrey,%
  urlcolor=myDarkBlue,%
  pdfstartview=Fit,%
  pdfview={XYZ null null null}%
}
%\hypersetup{                  		% beamer colors taken from elsewhere
%  hyperindex,%				% works with the beetle colour scheme
%  colorlinks%
%  linktocpage,%
%  plainpages=false,%
%  linkcolor=eddBlue,%
%  citecolor=eddDarkGrey,%
%  urlcolor=eddDarkBlue,%
%  pdfstartview=Fit,%
%  pdfview={XYZ null null null}%
%}

\RequirePackage{color}
\definecolor{Red}{rgb}{0.7,0,0}
\definecolor{myOrange}{rgb}{0.8,0.5,0.0}
\definecolor{myBlue}{rgb}{0.0,0.0,0.4}
\definecolor{myDarkBlue}{rgb}{0.1,0.1,0.4}
\definecolor{myDarkGrey}{rgb}{0.15,0.15,0.15}
% Doug's
\definecolor{Sinput}{rgb}{0,0,0.56}
\definecolor{Scode}{rgb}{0,0,0.56}
\definecolor{Soutput}{rgb}{0.56,0,0}
% 
\definecolor{Cmdinput}{rgb}{0,0,0.44}
\definecolor{Cmdoutput}{rgb}{0.44,0,0}
\definecolor{Cppinput}{rgb}{0.15,0.15,0.15}

%% from Doug, but mod'ed \R to use hyperref
\RequirePackage{fancyvrb}
\RequirePackage{xspace}
\RequirePackage{paralist}
% \newenvironment{Schunk}{\par\begin{minipage}{\textwidth}}{\end{minipage}}
% \DefineVerbatimEnvironment{Sinput}{Verbatim}{formatcom={\color{Sinput}},fontsize=\scriptsize}
% \DefineVerbatimEnvironment{Soutput}{Verbatim}{formatcom={\color{Soutput}},fontsize=\scriptsize}
% \DefineVerbatimEnvironment{Scode}{Verbatim}{formatcom={\color{Scode}},fontsize=\small}
% \DefineVerbatimEnvironment{Cmdinput}{Verbatim}{formatcom={\color{Cmdinput}},fontsize=\scriptsize}
% \DefineVerbatimEnvironment{Cmdoutput}{Verbatim}{formatcom={\color{Cmdoutput}},fontsize=\scriptsize}
% \DefineVerbatimEnvironment{Cppinput}{Verbatim}{formatcom={\color{Cppinput}},fontsize=\small}

% -- not \small 
\newcommand{\smallcode}[1]{{\color{Sinput}\small\texttt{#1}}}
\newcommand{\code}[1]{{\color{Sinput}\texttt{#1}}}
\newcommand{\Emph}[1]{\emph{\color{Scode}#1}}   
%\newcommand{\R}{\href{http://www.r-project.org}{\Emph{R}\xspace}}   %% ? sing \emph upsets beamer inside \href
\newcommand{\R}{\href{http://www.r-project.org}{\textsf{R}\xspace}}
\newcommand{\Rns}{\href{http://www.r-project.org}{\textsf{R}}}

% two old defintions
%\newcommand{\code}[1]{\texttt{#1}}
\newcommand{\screenshot}[1]{\centerline{\includegraphics[height=7.8cm,transparent]{#1}}}  % 7.8in


% If you have a file called "university-logo-filename.xxx", where xxx
% is a graphic format that can be processed by latex or pdflatex,
% resp., then you can add a logo as follows:
% NB transparent in Adobe but not in kpdf
% \pgfdeclareimage[height=0.6cm]{useR-logo}{figures/useR}
% \logo{\pgfuseimage{useR-logo}}

% \pgfdeclareimage[height=1cm]{debian-logo}{figures/debian-openlogo-100}
% \logo{\pgfuseimage{debian-logo}}

%% code from
%% http://old.nabble.com/Two-logos-instead-of-the-default-one-td16000584.html
%% and posted by Matthew Leingang
%\pgfdeclareimage[height=0.5cm]{msri-logo}{msri-logo-clipped} 
% put MSRI logo in bottom left 
%% \setbeamertemplate{sidebar left}{% 
%%    \vfill% 
%%    \rlap{\hskip0.1cm 
%%      \href{http://www.debian.org}{\pgfuseimage{debian-logo}}}% 
%%    %\vskip2pt% 
%%    %\llap{\usebeamertemplate***{navigation symbols}\hskip0.1cm}% 
%%    \vskip2pt% 
%% } 

%% from highlight
% Style definition file generated by highlight 2.7, http://www.andre-simon.de/ 

% Highlighting theme definition: 

% \newcommand{\hlstd}[1]{\textcolor[rgb]{0,0,0}{#1}}
% \newcommand{\hlnum}[1]{\textcolor[rgb]{0,0,0}{#1}}
% \newcommand{\hlesc}[1]{\textcolor[rgb]{0.74,0.55,0.55}{#1}}
% \newcommand{\hlstr}[1]{\textcolor[rgb]{0.74,0.55,0.55}{#1}}
% \newcommand{\hldstr}[1]{\textcolor[rgb]{0.74,0.55,0.55}{#1}}
% \newcommand{\hlslc}[1]{\textcolor[rgb]{0.67,0.13,0.13}{\it{#1}}}
% \newcommand{\hlcom}[1]{\textcolor[rgb]{0.67,0.13,0.13}{\it{#1}}}
% \newcommand{\hldir}[1]{\textcolor[rgb]{0,0,0}{#1}}
% \newcommand{\hlsym}[1]{\textcolor[rgb]{0,0,0}{#1}}
% \newcommand{\hlline}[1]{\textcolor[rgb]{0.33,0.33,0.33}{#1}}
% \newcommand{\hlkwa}[1]{\textcolor[rgb]{0.61,0.13,0.93}{\bf{#1}}}
% \newcommand{\hlkwb}[1]{\textcolor[rgb]{0.13,0.54,0.13}{#1}}
% \newcommand{\hlkwc}[1]{\textcolor[rgb]{0,0,1}{#1}}
% \newcommand{\hlkwd}[1]{\textcolor[rgb]{0,0,0}{#1}}
% \definecolor{bgcolor}{rgb}{1,1,1}




%%% Local Variables: 
%%% mode: latex
%%% TeX-master: "introhighperfR"
%%% End: 
