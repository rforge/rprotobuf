\documentclass[11pt]{article}
\usepackage{url}
\usepackage{vmargin}
\setpapersize{USletter}
\setmarginsrb{1in}{1in}{1in}{1in}{0pt}{0mm}{0pt}{0mm}
\usepackage{charter}

\newcommand{\proglang}[1]{\textsf{#1}}
\newcommand{\pkg}[1]{{\fontseries{b}\selectfont #1}}

\author{
Romain Fran\c{c}ois\\ {\small \url{romain@r-enthusiasts.com} } \and
Dirk Eddelbuettel\\ {\small \url{edd@debian.org} }
}

\title{\pkg{RProtoBuf}: Protocol Buffers for R }
\date{Submitted to \textsl{useR! 2010}}

\begin{document}

\maketitle
\thispagestyle{empty}
\begin{abstract}
  \addtolength{\parskip}{\baselineskip} 	% add a little vertical space
  \noindent % no ident for first paragraph
  %
  Protocol buffers are a flexible, efficient, automated mechanism for
  serializing structured data---think XML, but smaller, faster, and simpler.
  Users define how they want the data to be structured once in a
  \texttt{proto} file and then use special generated source code to easily
  write and read structured data to and from a variety of data streams and
  using a variety of officially supported languages--- \proglang{Java},
  \proglang{C++}, or \proglang{Python}---or third party implementations for
  languages such as \proglang{C\#}, \proglang{Perl}, \proglang{Ruby},
  \proglang{Haskell}, and now \proglang{R} via the \pkg{RProtoBuf} package.
  
  The \pkg{RProtoBuf} package implements \proglang{R} bindings to the
  \proglang{C++} protobuf library from Google. It uses features of the
  protocol buffer library to support creation, manipulation, parsing and
  serialization of protocol buffers messages. Taking advantage of facilities
  in the \pkg{Rcpp} package, \pkg{RProtoBuf} uses S4 classes and external
  pointers to expose objects that look and feel like standard \proglang{R}
  lists, but conforming to the language-agnostic definition of the message
  type.
  % [Dirk] what does that last half-sentence mean? constrained as in conforming
  % [Romain] yes. updated. I also wanted to say that the actual memory of the object
  %          is managed by the C++ library and not R, but I can't manage to spell it.  

  % [Dirk] ok to add 'forward-looking' statements?
  % [Romain] fine by me.
  As the protocal buffers library does not offer any built-in support
  for networked access to protocol buffer streams, we intend to take
  advantage on ongoing changes to the \proglang{R} system to expose a native
  \proglang{R} server. This is work-in-progress that we hope to report on at
  the conference.

  \noindent \textbf{Keywords:}  
  \proglang{R}, \proglang{C++}, Serialization, Data Interchange, Data Formats
\end{abstract}
\end{document}

%%% Local Variables: 
%%% mode: latex
%%% TeX-master: t
%%% End: 
