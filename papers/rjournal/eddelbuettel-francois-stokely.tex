% !TeX root = RJwrapper.tex
\title{RProtoBuf: Efficient Cross-Language Data Serialization in R}
\author{by Dirk Eddelbuettel, Romain Fran\c{c}ois, and Murray Stokely}

\maketitle

\abstract{Modern data collection and analysis pipelines often involve
 a sophisticated mix of applications written in general purpose and
 specialized programming languages.  Protocol Buffers are a popular
 method of serializing structured data between applications.  The
 \textbf{RProtoBuf} package provides a complete interface to this
 library.
TODO keep it less than 150 words.
}

\section{Introduction}

Comparison with what people start with in R : CSV
comparison with what is only slightly better: JSON

Introductory section which may include references in parentheses
\citep{R}, or cite a reference such as \citet{R} in the text.

Protocol buffers are a language-neutral, platform-neutral, extensible
way of serializing structured data for use in communications
protocols, data storage, and more.

Protocol Buffers offer key features such as an efficient data interchange
format that is both language- and operating system-agnostic yet uses a
lightweight and highly performant encoding, object serialization and
de-serialization as well data and configuration management. Protocol
buffers are also forward compatible: updates to the \texttt{proto}
files do not break programs built against the previous specification.

While benchmarks are not available, Google states on the project page that in
comparison to XML, protocol buffers are at the same time \textsl{simpler},
between three to ten times \textsl{smaller}, between twenty and one hundred
times \textsl{faster}, as well as less ambiguous and easier to program.

The protocol buffers code is released under an open-source (BSD) license. The
protocol buffer project (\url{http://code.google.com/p/protobuf/})
contains a C++ library and a set of runtime libraries and compilers for
C++, Java and Python.

With these languages, the workflow follows standard practice of so-called
Interface Description Languages (IDL)
(c.f. \href{http://en.wikipedia.org/wiki/Interface_description_language}{Wikipedia
  on IDL}).  This consists of compiling a protocol buffer description file
(ending in \texttt{.proto}) into language specific classes that can be used
to create, read, write and manipulate protocol buffer messages. In other
words, given the 'proto' description file, code is automatically generated
for the chosen target language(s). The project page contains a tutorial for
each of these officially supported languages:
\url{http://code.google.com/apis/protocolbuffers/docs/tutorials.html}

Besides the officially supported C++, Java and Python implementations, several projects have been
created to support protocol buffers for many languages. The list of known
languages to support protocol buffers is compiled as part of the
project page: \url{http://code.google.com/p/protobuf/wiki/ThirdPartyAddOns}

The protocol buffer project page contains a comprehensive
description of the language: \url{http://code.google.com/apis/protocolbuffers/docs/proto.html}

%This section may contain a figure such as Figure~\ref{figure:rlogo}.
%
%\begin{figure}[htbp]
%  \centering
%  \includegraphics{Rlogo}
%  \caption{The logo of R.}
%  \label{figure:rlogo}
%\end{figure}

\section{Dynamic use: Protocol Buffers and R}

This section describes how to use the R API to create and manipulate
protocol buffer messages in R, and how to read and write the
binary \emph{payload} of the messages to files and arbitrary binary
R connections.

\subsection{Importing proto files}

In contrast to the other languages (Java, C++, Python) that are officially
supported by Google, the implementation used by the \texttt{RProtoBuf}
package does not rely on the \texttt{protoc} compiler (with the exception of
the two functions discussed in the previous section). This means that no
initial step of statically compiling the proto file into C++ code that is
then accessed by R code is necessary. Instead, \texttt{proto} files are
parsed and processed \textsl{at runtime} by the protobuf C++ library---which
is much more appropriate for a dynamic language.

The \texttt{readProtoFiles} function allows importing \texttt{proto}
files in several ways.

% Example code snippet.
% TODO(mstokely): Remove this.
\begin{example}
  x <- 1:10
  result <- myFunction(x)
\end{example}

\section{Related work on IDLs (greatly expanded from what you have)}

\section{Design tradeoffs: reflection vs proto compiler (not addressed
  at all in current vignettes)}

\subsection{Performance considerations}

TODO RProtoBuf is quite flexible and easy to use for interactive
analysis, but it is not designed for certain classes of operations one
might like to do with protocol buffers.  For example, taking a list of
10,000 protocol buffers, extracting a named field from each one, and
computing a aggregate statistics on those values would be extremely
slow with RProtoBuf, and while this is a useful class of operations,
it is outside of the scope of RProtoBuf.  We should be very clear
about this to clarify the goals and strengths of RProtoBuf and its
reflection and object mapping.

\subsection{Serialization comparison}

TODO comparison of protobuf serialization sizes/times for various vectors.  Compared to R's native serialization.  Discussion of the RHIPE approach of serializing any/all R objects, vs more specific protocol buffers for specific R objects.

\section{Basic usage example - tutorial.Person}

\section{Application: distributed Data Collection with MapReduce}

We could describe a common MapReduce pattern of having the MR written
in another language output protocol buffers that are later pulled into
R.  There is some text about this in section 2 of
http://cran.r-project.org/web/packages/HistogramTools/vignettes/HistogramTools.pdf 

\section{Application: Sending/receiving Interaction With Servers}

\section{Summary}

This file is only a basic article template. For full details of \emph{The R Journal} style and information on how to prepare your article for submission, see the \href{http://journal.r-project.org/latex/RJauthorguide.pdf}{Instructions for Authors}.

\bibliography{eddelbuettel-francois-stokely}

\address{Author One\\
  Affiliation\\
  Address\\
  Country}
\email{author1@work}

\address{Author Two\\
  Affiliation\\
  Address\\
  Country}
\email{author2@work}

\address{Murray Stokely\\
  Google, Inc.\\
  1600 Amphitheatre Parkway\\
  Mountain View, CA 94043\\
  USA}
\email{mstokely@google.com}
