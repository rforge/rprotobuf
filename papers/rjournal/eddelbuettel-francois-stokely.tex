% !TeX root = RJwrapper.tex
\title{RProtoBuf: Efficient Cross-Language Data Serialization in R}
\author{by Dirk Eddelbuettel, Romain Fran\c{c}ois, and Murray Stokely}

\maketitle

\abstract{Modern data collection and analysis pipelines often involve
 a sophisticated mix of applications written in general purpose and
 specialized programming languages.  Protocol Buffers are a popular
 method of serializing structured data between applications---while remaining
 independent of programming languages or operating system.  The
 \CRANpkg{RProtoBuf} package provides a complete interface to this
 library.
 %TODO(ms) keep it less than 150 words.
}

%TODO(de) 'protocol buffers' or 'Protocol Buffers' ?

\section{Introduction}

Modern data collection and analysis pipelines are increasingly being
built using collections of components to better manage software
complexity through reusability, modularity, and fault
isolation \citep{Wegiel:2010:CTT:1932682.1869479}.  Different
programming languages are often used for the different phases of data
analysis -- collection, cleaning, analysis, post-processing, and
presentation in order to take advantage of the unique combination of 
performance, speed of development, and library support offered by
different environments.  Each stage of the data
analysis pipeline may involve storing intermediate results in a
file or sending them over the network.  Programming langauges such as
Java, Ruby, Python, and R include built-in serialization support, but
these formats are tied to the specific programming language in use.
CSV files can be read and written by many applications and so are
often used for exporting tabular data.  However, CSV files have a
number of disadvantages, such as a limitation of exporting only
tabular datasets, lack of type-safety, inefficient text representation
and parsing, and abiguities in the format involving special
characters.  JSON is another widely supported format used mostly on
the web that removes many of these disadvantages, but it too suffers
from being too slow to parse and also does not provide strong typing
between integers and floating point.  Large numbers of JSON messages
would also be required to duplicate the field names with each message.

This article describes the basics of Google's Protocol Buffers through
an easy to use R package, \CRANpkg{RProtoBuf}.  After describing the
basics of protocol buffers and \CRANpkg{RProtoBuf}, we illustrate
several common use cases for protocol buffers in data analysis.

\section{Protocol Buffers}

Once the data serialization needs get complex enough, application
developers typically benefit from the use of an \emph{interface
description language}, or \emph{IDL}.  IDLs like Google's Protocol
Buffers and Apache Thrift provide a compact well-documented schema for
cross-langauge data structures as well efficient binary interchange
formats.  The schema can be used to generate model classes for
statically typed programming languages such as C++ and Java, or can be
used with reflection for dynamically typed programming languages.
Since the schema is provided separately from the encoded data, the
data can be efficiently encoded to minimize storage costs of the
stored data when compared with simple ``schema-less'' binary
interchange formats like BSON.

%BSON, msgpack, Thrift, and Protocol Buffers take this latter approach,
%with the

% There are references comparing these we should use here.

TODO Also mention Thrift and msgpack and the references comparing some
of these tradeoffs.

Introductory section which may include references in parentheses
\citep{R}, or cite a reference such as \citet{R} in the text.

Protocol buffers are a language-neutral, platform-neutral, extensible
way of serializing structured data for use in communications
protocols, data storage, and more.

Protocol Buffers offer key features such as an efficient data interchange
format that is both language- and operating system-agnostic yet uses a
lightweight and highly performant encoding, object serialization and
de-serialization as well data and configuration management. Protocol
buffers are also forward compatible: updates to the \texttt{proto}
files do not break programs built against the previous specification.

While benchmarks are not available, Google states on the project page that in
comparison to XML, protocol buffers are at the same time \textsl{simpler},
between three to ten times \textsl{smaller}, between twenty and one hundred
times \textsl{faster}, as well as less ambiguous and easier to program.

The protocol buffers code is released under an open-source (BSD) license. The
protocol buffer project (\url{http://code.google.com/p/protobuf/})
contains a C++ library and a set of runtime libraries and compilers for
C++, Java and Python.

With these languages, the workflow follows standard practice of so-called
Interface Description Languages (IDL)
(c.f. \href{http://en.wikipedia.org/wiki/Interface_description_language}{Wikipedia
  on IDL}).  This consists of compiling a protocol buffer description file
(ending in \texttt{.proto}) into language specific classes that can be used
to create, read, write and manipulate protocol buffer messages. In other
words, given the 'proto' description file, code is automatically generated
for the chosen target language(s). The project page contains a tutorial for
each of these officially supported languages:
\url{http://code.google.com/apis/protocolbuffers/docs/tutorials.html}

Besides the officially supported C++, Java and Python implementations, several projects have been
created to support protocol buffers for many languages. The list of known
languages to support protocol buffers is compiled as part of the
project page: \url{http://code.google.com/p/protobuf/wiki/ThirdPartyAddOns}

The protocol buffer project page contains a comprehensive
description of the language: \url{http://code.google.com/apis/protocolbuffers/docs/proto.html}

%This section may contain a figure such as Figure~\ref{figure:rlogo}.
%
%\begin{figure}[htbp]
%  \centering
%  \includegraphics{Rlogo}
%  \caption{The logo of R.}
%  \label{figure:rlogo}
%\end{figure}

\section{Dynamic use: Protocol Buffers and R}

TODO(ms): random citations to work in:

We make use of Object Tables \citep{RObjectTables} for lookup.
Many sources compare data serialization formats and show protocol
buffers very favorably to the alternatives, such
as \citep{Sumaray:2012:CDS:2184751.2184810}

This section describes how to use the R API to create and manipulate
protocol buffer messages in R, and how to read and write the
binary \emph{payload} of the messages to files and arbitrary binary
R connections.

\subsection{Importing proto files}

In contrast to the other languages (Java, C++, Python) that are officially
supported by Google, the implementation used by the \texttt{RProtoBuf}
package does not rely on the \texttt{protoc} compiler (with the exception of
the two functions discussed in the previous section). This means that no
initial step of statically compiling the proto file into C++ code that is
then accessed by R code is necessary. Instead, \texttt{proto} files are
parsed and processed \textsl{at runtime} by the protobuf C++ library---which
is much more appropriate for a dynamic language.

The \texttt{readProtoFiles} function allows importing \texttt{proto}
files in several ways.

% Example code snippet.
% TODO(mstokely): Remove this.
\begin{example}
  x <- 1:10
  result <- myFunction(x)
\end{example}

\section{Related work on IDLs (greatly expanded from what you have)}

\section{Design tradeoffs: reflection vs proto compiler (not addressed
  at all in current vignettes)}

\subsection{Performance considerations}

TODO RProtoBuf is quite flexible and easy to use for interactive
analysis, but it is not designed for certain classes of operations one
might like to do with protocol buffers.  For example, taking a list of
10,000 protocol buffers, extracting a named field from each one, and
computing a aggregate statistics on those values would be extremely
slow with RProtoBuf, and while this is a useful class of operations,
it is outside of the scope of RProtoBuf.  We should be very clear
about this to clarify the goals and strengths of RProtoBuf and its
reflection and object mapping.

\subsection{Serialization comparison}

TODO comparison of protobuf serialization sizes/times for various vectors.  Compared to R's native serialization.  Discussion of the RHIPE approach of serializing any/all R objects, vs more specific protocol buffers for specific R objects.

\section{Basic usage example - tutorial.Person}

\section{Application: distributed Data Collection with MapReduce}

We could describe a common MapReduce pattern of having the MR written
in another language output protocol buffers that are later pulled into
R.  There is some text about this in section 2 of
http://cran.r-project.org/web/packages/HistogramTools/vignettes/HistogramTools.pdf 

\section{Application: Sending/receiving Interaction With Servers}

\section{Summary}

This file is only a basic article template. For full details of \emph{The R Journal} style and information on how to prepare your article for submission, see the \href{http://journal.r-project.org/latex/RJauthorguide.pdf}{Instructions for Authors}.

\bibliography{eddelbuettel-francois-stokely}

\address{Dirk Eddelbuettel\\
  Debian and R Projects\\
  711 Monroe Avenue, River Forest, IL 60305\\
  USA}
\email{edd@debian.org}

\address{Author Two\\
  Affiliation\\
  Address\\
  Country}
\email{author2@work}

\address{Murray Stokely\\
  Google, Inc.\\
  1600 Amphitheatre Parkway\\
  Mountain View, CA 94043\\
  USA}
\email{mstokely@google.com}
