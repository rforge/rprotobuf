
\documentclass[10pt]{article}
\usepackage{url}
\usepackage{vmargin}
\setpapersize{USletter}
% left top right bottom -- headheight headsep footheight footskop
\setmarginsrb{1in}{1in}{1in}{0.5in}{0pt}{0mm}{10pt}{0.5in}
\usepackage{charter}

\setlength{\parskip}{1ex plus1ex minus1ex}
\setlength{\parindent}{0pt}

\newcommand{\proglang}[1]{\textsf{#1}}
\newcommand{\pkg}[1]{{\fontseries{b}\selectfont #1}}

\newcommand{\pointRaised}[2]{\smallskip %\hrule 
  \textsl{{\fontseries{b}\selectfont #1}: #2}\newline}
\newcommand{\simplePointRaised}[1]{\bigskip \hrule\textsl{#1} }
\newcommand{\reply}[1]{\textbf{Reply}:\ #1 \smallskip } %\hrule \smallskip}

\begin{document}

\author{Dirk Eddelbuettel\\Debian Project \and 
        Murray Stokely\\Google, Inc \and
        Jeroen Ooms\\UCLA}
\title{Submission JSS 1313: \\ Response to Reviewers' Comments}
\maketitle 
\thispagestyle{empty}

Thank you for reviewing our manuscript, and for giving us an opportunity to
rewrite, extend and and tighten both the paper and the underlying package.

\smallskip
We truly appreciate the comments and suggestions. Below, we have regrouped the sets
of comments, and have provided detailed point-by-point replies.
%
We hope that this satisfies the request for changes necessary to proceed with
the publication of the revised and updated manuscript, along with the revised
and updated package (which was recently resubmitted to CRAN as version 0.4.2).

\section*{Response to Reviewer \#1}

\pointRaised{Comment 1}{Overall, I think this is a strong paper. Cross-language communication
  is a challenging problem, and good solutions for R are important to
  establish R as a well-behaved member of a data analysis pipeline. The
  paper is well written, and I recommend that it be accepted subject to
  the suggestions below.}
\reply{Thank you. We are providing a point-by-point reply below.}

\subsubsection*{More big picture, less details}

\pointRaised{Comment 2}{Overall, I think the paper provides too much detail on
  relatively unimportant topics and not enough on the reasoning behind
  important design decisions. I think you could comfortably reduce the paper
  by 5-10 pages, referring the interested reader to the documentation for
  more detail.}
\reply{The paper is now six pages shorter at just 23 pages.
  Sections 3 - 8 (all but Section 1 (``Introduction''), Section 2 (``Protocol Buffers''),
  and Section 9 (``Conclusion'') have been thoroughly rewritten to address the specific and
  general feedback in these reviews.}

\pointRaised{Comment 3}{I'd recommend shrinking section 3 to ~2 pages, and removing the
  subheadings. This section should quickly orient the reader to the
  RProtobuf API so they understand the big picture before learning more
  details in the subsequent sections. I'd recommend picking one OO style
  and sticking to it in this section - two is confusing.}
\reply{We followed this recommendation, reduced section 3 to about
  $2\frac{1}{2}$ pages, removed the subheadings and tightened the exposition.}

\pointRaised{Comment 3}{Section 4 dives into the details without giving a good overview and
  motivation. Why use S4 and not RC? How are the objects made mutable?
  Why do you provide both generic function and message passing OO
  styles? What does \$ do in this context? What the heck is a
  pseudo-method? Spend more time on those big issues rather than
  describing each class in detail. Reduce class descriptions to a
  bulleted list giving a high-level overview, then encourage the reader
  to refer to the documentation for further details. Similarly, Tables
  3-5 belong in the documentation, not in a vignette/paper.}
\reply{Done. RProtoBuf was designed and implemented before RC were
  available, and this is now noted explicitly in a new footnote.  Explanation of how
  they are made mutable has been added.  Better explanation of the
  two styles and '\$' as been added.  We are no longer using the
  confusing term 'pseudo-method' anywhere.  We also moved Tables 3-5 into the
  documentation and out of the paper, as suggested.}

\pointRaised{Comment 4}{Section 7 is weak. I think the important message is that RProtobuf is
  being used in practice at large scale for for large data, and is
  useful for communicating between R and Python. How can you make that
  message stronger while avoiding (for the purposes of this paper) the
  relatively unimportant details of the map-reduce setup?}
\reply{Done.  Rewritten with more motivation taking into account this feedback.}

\subsubsection*{R to/from Protobuf translation}

\pointRaised{Comment 5}{The discussion of R to/from Protobuf could be improved. Table 9 would be
  much simpler if instead of Message, you provided a "vectorised"
  Messages class (this would also make the interface more consistent and
  hence the package easier to use).}
\reply{This is a good observation that only became clear to us after
  significant usage of \texttt{RProtoBuf}.  Providing a full ``vectorized'' Messages class would require slicing
  operators that let you quickly extract a given field from each
  element of the message vector in order to be really useful.  This
  would require significant amounts of C++ code for efficient
  manipulation on the order of data.table or other similar large C++ R
  packages on CRAN.  There is another package called Motobuf by other authors
  that takes this approach but in practice (at least for the several hundred
  users at Google), the ease-of-use provided by the simple Message interface of RProtoBuf
  has won with users.  It is still future work to keep the simple
  interactive interface of RProtoBuf with the vectorized efficiency of
  Motobuf.  For now, users typically do their slicing of vectors like
  this through a distributed database (NewSQL is the term of the day?)
  like Dremel or other system and then just get the response Protocol
  Buffers in return to the request.}

\pointRaised{Comment 6}{Along these lines, I think it would make sense to combine sections 5
  and 6 and discuss translation challenges in both direction
  simultaneously. At the minimum, add the equivalent for Table 9 that
  shows how important R classes are converted to their protobuf
  equivalents.}
\reply{Done. We have updated these sections to make it clearer that the main
  distinction is between schema-based datastructures (Section 5) and
  schema-less use where a catch-all \texttt{.proto} is used (Section 6).
  Neither section is meant to focus on only a single direction of the
  conversion, but how conversion works when you have a schema or not.
  How important R classes are converted to their protobuf equivalents
  isn't super useful as a C++, Java, or Python program is unlikely to
  want to read in an R data.frame exactly as it is defined.  Much more
  likely is an application-specific message format is defined between the
  two services, such as the HistogramTools example in the next section.
  Much more detail has been added to an interesting part of section 6 --
  which datasets exactly are better served with RProtoBuf than
  \texttt{base::serialize} and why?}

\pointRaised{Comment 7}{You should discuss how missing values are handled for strings and
  integers, and why enums are not equivalent to factors. I think you
  could make explicit how coercion of factors, dates, times and matrices
  occurs, and the implications of this on sharing data structures
  between programming languages. For example, how do you share date/time
  data between R and python using RProtoBuf?}
\reply{All of these details are application-specific, whereas
  RProtoBuf is an infrastructure package.  Distributed systems define
  their own interfaces, with their own date/time fields, usually as
  a double of fractional seconds since the unix epoch for the systems I
  have worked on.  An example is given for Histograms in the next
  section.  Factors could be represented as repeated enums in Protocol
  Buffers, certainly, if that is how one wanted to define a schema.}

\pointRaised{Comment 8}{Table 10 is dying to be a plot, and a natural companion would be to
  show how long it takes to serialise data frames using both RProtoBuf
  and R's native serialisation. Is there a performance penalty to using
  protobufs?}
\reply{Done. Table 10 has been replaced with a plot, the outliers are
  labeled, and the text now includes some interesting explanation
  about the outliers.  Page 4 explains that the R implementation of
  Protocol Buffers uses reflection to make operations slower but makes
  it more convenient for interactive data analysis.  None of the
  built-in datasets are large enough for performance to really come up
  as an issue, and for any serialization method examples could be
  found that significantly favor one over another in runtime, so we
  don't think there will be benefit to adding anything here.  }

\subsubsection*{RObjectTables magic}

\pointRaised{Comment 9}{The use of RObjectTables magic makes me uneasy. It doesn't seem like a
  good fit for an infrastructure package and it's not clear what
  advantages it has over explicitly loading a protobuf definition into
  an object.}
\reply{Done. More information about the advantages and disadvantages of this
  approach have been added.}

\pointRaised{Comment 10}{Using global state makes understanding code much harder. In Table 1,
  it's not obvious where \texttt{tutorial.Person} comes from. Is it loaded by
  default by RProtobuf? This need some explanation. In Section 7, what
  does \texttt{readProtoFiles()} do? Why does \texttt{RProtobuf} need to be attached
  as well as \texttt{HistogramTools}? This needs more explanation, and a
  comment on the implications of this approach on CRAN packages and
  namespaces.}
\reply{Done. We followed this recommendation and added explanation for how
  \texttt{tutorial.Person} is loaded, specifically : \emph{A small number of message types are imported when the
    package is first loaded, including the tutorial.Person type we saw in
    the last section.}  Thank you also for spotting the superfluous attach
  of \texttt{RProtoBuf}, it has been removed from the example.}

\pointRaised{Comment 11}{
  I'd prefer you eliminate this magic from the magic, but failing that,
  you need a good explanation of why.}
\reply{Done. We've added more explanation about this.}

\subsubsection*{Code comments}

\pointRaised{Comment 12}{Using \texttt{file.create()} to determine the absolute path seems like a bad idea.}
\reply{Done. We followed this recommendation and removed two instances of
  \texttt{file.create()} for this purpose with calls to
  \texttt{normalizePath} with \texttt{mustWork=FALSE}.}

\subsubsection*{Minor niggles}

\pointRaised{Comment 13}{Don't refer to the message passing style of OO as traditional.}
\reply{Done. We don't refer to this style as traditional anywhere in
  the manuscript anymore.}

\pointRaised{Comment 14}{In Section 3.4, if messages isn't a vectorised class, the default
   print method should use \texttt{cat()} to eliminate the confusing \texttt{[1]}.}
\reply{Done, thanks.}

\pointRaised{Comment 15}{The REXP definition would have been better defined using an enum that
   matches R's SEXPTYPE "enum". But I guess that ship has sailed.}
\reply{Acknowledged.  We chose to maintain compatibility with RHIPE here.  The main
use of RProtoBuf is not with \texttt{rexp.proto} however -- it with
application-specific schemas in \texttt{.proto} files for sending data between
applications.  Users that want to do something very R-specific are
welcome to use their own \texttt{.proto} files with an enum to represent R SEXPTYPEs.}

\pointRaised{Comment 16}{Why does \texttt{serialize\_pb(CO2, NULL)} fail silently? Shouldn't it at least
   warn that the serialization is partial?}
\reply{Done. We fixed this and \texttt{serialize\_pb} now works for all built-in datatypes in R
  and no longer fails silently if it encounters something it can't serialize.}

\section*{Response to Reviewer \#2}

\pointRaised{Comment 1}{The paper gives an overview of the RProtoBuf package which implements an 
  R interface to the Protocol Buffers library for an efficient 
  serialization of objects. The paper is well written and easy to read. 
  Introductory code is clear and the package provides objects to play with 
  immediately without the need to jump through hoops, making it appealing. 
  The software implementation is executed well.}
\reply{Thank you.}

\pointRaised{Comment 2}{There are, however, a few inconsistencies in the implementation and some 
  issues with specific sections in the paper. In the following both issues 
  will be addressed sequentially by their occurrence in the paper.}
\reply{Done. These and others have been identified and addressed.  Thank you
  for taking the time to enumerate these issues.}

\pointRaised{Comment 3}{p.4 illustrates the use of messages. The class implements list-like 
  access via \texttt{[[} and \$, but strangely \texttt{names()} return NULL and \texttt{length() }
  doesn't correspond to the number of fields leading to startling results like
the following:}

\begin{verbatim}
 > p
[1] "message of type 'tutorial.Person' with 2 fields set"
 > length(p)
[1] 2
 > p[[3]]
[1] ""
\end{verbatim}
\reply{Done. We have corrected the list-like accessor, fixed \texttt{length()} to
  correspond to the number of set fields, and added \texttt{names()}:}
\begin{verbatim}
> p
message of type 'tutorial.Person' with 0 fields set
> length(p)
[1] 0
> p[[3]]
[1] ""
> p$id <- 1
> length(p)
[1] 1
> names(p)
[1] "name"  "id"    "email" "phone"
\end{verbatim}

\pointRaised{Comment 3 cont.}{The inconsistencies get even more bizarre with descriptors (p.9):}

\begin{verbatim}
 > tutorial.Person$email
[1] "descriptor for field 'email' of type 'tutorial.Person' "
 > tutorial.Person[["email"]]
Error in tutorial.Person[["email"]] : this S4 class is not subsettable
 > names(tutorial.Person)
NULL
 > length(tutorial.Person)
[1] 1
\end{verbatim}
\reply{Done. We agree, and have addressed this inconsistency.  Thank you for
  catching this.}
\begin{verbatim}
> tutorial.Person$email
descriptor for field 'email' of type 'tutorial.Person' 
> tutorial.Person[["email"]]
descriptor for field 'email' of type 'tutorial.Person' 
> names(tutorial.Person)
[1] "name"        "id"          "email"       "phone"       "PhoneNumber"
[6] "PhoneType"  
> length(tutorial.Person)
[1] 6
\end{verbatim}

\pointRaised{Comment 3 cont.}{It appears that there is no way to find out the fields of a descriptor 
  directly (although the low-level object methods seem to be exposed as 
  \texttt{\$field\_count()} and \texttt{\$fields()} - but that seems extremely cumbersome). 
  Again, implementing names() and subsetting may help here.}
\reply{Done. We have implemented \texttt{names} and subsetting.  Thank you for the
  suggestion.}
\begin{verbatim}
> tutorial.Person[[1]]
descriptor for field 'name' of type 'tutorial.Person' 
> tutorial.Person[[2]]
descriptor for field 'id' of type 'tutorial.Person' 
\end{verbatim}

\pointRaised{Comment 4}{Another inconsistency concerns the \texttt{as.list()} method which by design 
  coerces objects to lists (see \texttt{?as.list}), but the implementation for 
  EnumDescriptor breaks that contract and returns a vector instead:}

\begin{verbatim}
 > is.list(as.list(tutorial.Person$PhoneType))
[1] FALSE
 > str(as.list(tutorial.Person$PhoneType))
  Named int [1:3] 0 1 2
  - attr(*, "names")= chr [1:3] "MOBILE" "HOME" "WORK"
\end{verbatim}

\reply{Done, thank you. New output below:}
\begin{verbatim}
> is.list(as.list(tutorial.Person$PhoneType))
[1] TRUE
> str(as.list(tutorial.Person$PhoneType))
List of 3
 $ MOBILE: int 0
 $ HOME  : int 1
 $ WORK  : int 2
\end{verbatim}

\pointRaised{Comment 4 cont}{As with the other interfaces, names() returns NULL so it is again quite
  difficult to perform even simple operations such as finding out the 
  values. It may be natural use some of the standard methods like names(), 
  levels() or similar. As with the previous cases, the lack of [[ support
  makes it impossible to map named enum values to codes and vice-versa.}
\reply{Done, thank you.  New output:}
\begin{verbatim}
> names(tutorial.Person$PhoneType)
[1] "MOBILE" "HOME"   "WORK"  
> tutorial.Person$PhoneType[["HOME"]]
[1] 1
\end{verbatim}

\pointRaised{Comment 5}{In general, the package would benefit from one pass of checks to assess
  the consistency of the API. Since the authors intend direct interaction
  with the objects via basic standard R methods, the classes should behave 
  consistently.}
\reply{We made several passes, correcting issues as documented in the
  \texttt{ChangeLog} and now present in our latest 0.4.2 release on CRAN.}

\pointRaised{Comment 6}{Finally, most classes implement coercion to characters, which is not 
  mentioned and is not quite intuitive for some objects. For example, one
  may think that \texttt{as.character()} on a file descriptor returns let's say the 
  filename, but we get:}

\begin{verbatim}
 > cat(as.character(tutorial.Person$fileDescriptor()))
syntax = "proto2";

package tutorial;

option java_package = "com.example.tutorial";
option java_outer_classname = "AddressBookProtos";
[...]
\end{verbatim}
\reply{The behavior is documented in the package documentation but
  seemed like a minor detail not important for an already-long paper.
  In choosing the debug output for a file descriptor we agree
  that \texttt{filename} is a reasonable thing to expect, but we also
  think that the contents of the \texttt{.proto} file is also
  reasonable, but more useful.  We document this in the help for
  ``FileDescriptor-class'', the vignette, and other sources.
  \texttt{@filename} is one of the slots of the FileDescriptor class
  and so very easy to find.  The contents of the \texttt{.proto} are
  not as easily accessible in a slot, however, and so we find it much
  more useful to be output with \texttt{as.character()}.}

\pointRaised{Comment 7}{It is not necessary clear what java\_package has to do with a file 
  descriptor in R. Depending on the intention here, it may be useful to 
  explain this feature.
}
\reply{Done. This snippet has been removed as part of the general move of
  less relevant details to the package documentation, but for
  reference the \texttt{.proto} file syntax is defined in the Protocol Buffers
  language guide which is referenced earlier. It is a cross platform
  library and so this syntax specifies some parameters when Java code
  is used to access the structures defined in this file.  No such
  special syntax is required in the \texttt{.proto} files for R
  language code and so this line about java\_package was not relevant
  or needed in any way for RProtoBuf and is documented elsewhere.}

\subsubsection*{Other comments:}

\pointRaised{Comment 8}{p.17: "does not support ... function, language or environment. Such 
  objects have no native equivalent type in Protocol Buffers, and have 
  little meaning outside the context or R"
  That is certainly false. Native mirror of environments are hash tables - 
  a very useful type indeed. Language objects are just lists, so there is
  no reason to not include them - they can be useful to store expressions
  that may not be necessary specific to R. Further on p. 18 your run into
  the same problem that could be fixed so easily.}
\reply{Acknowledged.  Environments are more than just hash
  tables because they include other configuration parameters that are
  necessary to serialize as well to make sure
  serialization/unserialization is indempotent, but we agree it is
  cleaner and the package and the exposition in the paper to just make
  sure we serialize everything.  We can now fall back to
  \texttt{base::serialize()} and storing the bits in a rawString type of
  RProtoBuf to make the R schema-less serialization more complete.}

\pointRaised{Comment 9}{The examples in sections 7 and 8 are somewhat weak. It does not seem 
  clear why one would wish to unleash the power of PB just to transfer 
  breaks and counts for plotting - even a simple ASCII file would do that
  just fine. The main point in the example is presumably that there are 
  code generation methods for Hadoop based on PB IDL such that Hadoop can
  be made aware of the data types, thus making a histogram a proper record 
  that won't be split, can be combined etc. -- yet that is not mentioned 
  nor a way presented how that can be leveraged in practice. The Python 
  example code simply uses a static example with constants to simulate the 
  output of a reducer so it doesn't illustrate the point - the reader is 
  left confused why something as trivial would require PB while a savvy 
  reader is not able to replicate the illustrated process. Possibly 
  explaining the benefits and providing more details on how one would 
  write such a job would make it much more relevant.}
\reply{Yes, we added more detail about the advantages of using a
  proper data type for the histograms in this example that you mentioned here -- the
  ability to write combiners, prevent arbitrary splitting of the
  records, etc that can greatly improve performance.  We agree with
  the other reviewer that we don't want to get bogged down in details
  about a particular MapReduce implementation (such as Hadoop) and so
  now we specifically mention that goal here.
  I think we make a better connection now between the
  abstract MapReduce example given, and then the simpler Python
  example code with a static example.}

\pointRaised{Comment 10}{Section 8 is not very well motivated. It is much easier to use other 
  formats for HTTP exchange - JSON is probably the most popular, but even
  CSV works in simple settings. PB is a much less common standard. The 
  main advantage of PB is the performance over the alternatives, but HTTP
  services are not necessarily known for their high-throughput so why one
  would sacrifice interoperability by using PB (they are still more hassle 
  and require special installations)? It would be useful if the reason 
  could be made explicit here or a better example chosen.}
\reply{Done. This section has been reworded to make it shorter and more
  crisp, with fewer extraneous details about OpenCPU. Protocol
  Buffers is an efficient protocol used between distributed systems at
  many of the world's largest internet companies (Twitter, Sony,
  Google, etc.) but the design and implementation of a large
  enterprise-scale distributed system with a complex RPC system and
  serialization needs is well beyond the scope of what we can add to a
  paper about RProtoBuf.  We chose this example because it is a much
  more accessible example that any reader can use to easily
  send/receive RPCs and parse the results with RProtoBuf.}

\end{document}

%%% Local Variables: 
%%% mode: latex
%%% TeX-master: t
%%% End: 
